\documentclass[12pt]{article}
\usepackage[left=3cm,right=3cm,top=2cm,bottom=2cm]{geometry} % page settings
\usepackage{amsmath} % provides many mathematical environments & tools
\newtheorem{theorem}{Theorem}
\usepackage{amsfonts}
\usepackage{amssymb}
\usepackage{titlesec}
\usepackage{multirow}

\usepackage{tikz}
\usetikzlibrary{graphdrawing,graphs} 
\usegdlibrary {layered}

\usetikzlibrary{arrows,shapes.gates.logic.US,shapes.gates.logic.IEC,calc}

\begin{document}

\title{\large \textbf{Introduction to Computer Systems Coursework 2 \\ 
\vspace{0.5em}
\large Combinatorial Circuit Design}}


\author{Mateusz Lichota}
\date{\today}
\maketitle

\section {Problem specification}
As per the list supplied on CATe my number is $DECNUM_{10}$, which is $B4NUM$ in base four.

Looking it up in the table I obtained the following specification for my circuit:

\begin{table}[h]
  \centering
  \begin{tabular}{|c|c|c|c|}
  \hline  
  
  %SPECTABLE

  \end{tabular}
\end{table}

From this specification a truth table can be constructed:

\begin{table}[h]
  \centering
\begin{tabular}{|c|c|c|c|c|c|}
  \hline


  %TRUTHTABLE


  \hline
\end{tabular}
\end{table}

Now, while I could use a Karnaugh map to represent the truth table, and find a circuit implementing it,
the produced circuit would (A) only minimize the total number of logic gates, instead of considering
the actual relative sizes of the gates, and (B) would not even be close to optimal even in terms of the
number of gates used. Therefore, instead of using a Karnaugh map I decided to implement a MASSIVE brute
force search program in haskell to find a circuit close to optimal for my specification.

\section {Brute force search}

\subsection { The search space }
When conducting a brute force search it is important to precisely describe the search space.
Here we are looking for a cost-minimizing circuit consisting of a number of 2-input, 1-output logic gates:
(And, Nand, Or, Nor, Xor, Xnor), and possibly some inverters. This circuit should have 4 inputs: ($C_1$, $C_0$, $A$, $B$) 
and one output (R). It should also not contain any cycles. The cost of a circuit is defined as the total size of
all the logic gates the circuit uses, as per the table supplied in the exercise specification:

\begin{table}[h]
  \centering
  \begin{tabular}{|c|c|}
  \hline  
  
  Gate & Nominal Size \\
  \hline
  INVERTER & 3 \\
  NAND, NOR & 4 \\
  AND, OR & 6 \\
  XOR, XNOR & 8 \\
  \hline

  \end{tabular}
\end{table}

\subsection {The use of inverters}
Since the set of all gates that we are going to be using is
closed under output inversion, using an inverter on the output of some gate will always be
more expensive than using the output-inverted gate (e.g. inverting the output of a NAND gate costs
3+4=7, but simply using an AND gate costs 6). Therefore, the only way it might make sense to use
inverters is to invert one of the inputs (e.g. $C_0$) straight away. This is the first place where
I cut corners: in order to restrict the problem to 2-input gates I would have to treat every inverted
input as an essentially separate input, thus making the nubmer of inputs to the circuit variable.
Henc, I've decided to only ever either not invert an input, or invert it straight away and not use the
uninverted input at all. This way we always have four inputs, each of which is either inverted or not.

\subsection { Treating circuits as DAGs}

Formally, our circuit is a directed acyclic graph (DAG) with the following properties:
\begin {itemize}
  \item There are four source nodes with indegree 0 - the (possibly inverted) inputs.
  \item There is exactly one sink node with outdegree 0 and indegree 2 - the last gate.
  \item All other nodes have indegree 2 and outdegree 1 - the remaining gates.
\end {itemize}

Obviously, there is an infinite number of such DAGs, as they could be arbitrarily large. However,
graphs with a large number of nodes (logic gates) are not likely to be optimal (as each gate has a positive cost).
Furthermore, the experiments the I describe later show that exhaustively searching all n-gate circuits
is only feasible for $n<=6$, which sort of settles the issue of how far to search for me.

 
\subsection { Generating circuits }

In order to generate all circuits I first generate all generic DAGs with the properties described
above, and then I create all circuits based on every single DAG by substituting all nodes with every
possible combination of gates. 

\subsubsection {Generating DAGs}

In order to generate all n-gate DAGs I devised a recursive algorithm. The algorithm is as follows:

\begin {enumerate}
  \item Assume that we have four input nodes ($C_1, C_0, A, B$), and $n$ 'gate' nodes \\ ($N_1, N_2, \ldots, N_n$),
  which are sorted by the order in which they will appear in the topological sort of the completed graph.
  
  \item Start from node $N_n$. Assume we are given a list of all nodes $N_j : j < n$ which are not yet used as
  inputs to another nodes. Generate all possible pairs of inputs to $N_n$. Those pairs can be unordered, 
  since all gates we are using are input-permutation-invariant. The elements of the pair can be $C_1 C0, A, B$ or
  any of the nodes $N_1, \ldots N_{n-1}$. If the node $N_{n-1}$ is unused, it has to be one of the inputs to $N_n$,
  so it has to be an element of all of the generated pairs.

  \item For each pair of inputs to $N_n$, call the algorithm recursively on $N_{n-1}$, supplying the list of unused
  nodes with the two elements of the pair removed.
\end{enumerate}

This is not an algorithm coursework, so the description of the algorithm given here is not super precise and doesn't
include any edge cases or optimizations, but the actual haskell program implementing this algorithm is available on my github,
and I have verified that it indeed generates all n-gate DAGs for $n <= 4$ manually.

\subsection {Generating circuits from DAGs}

Given a generic n-gate DAG we can generate $6^n$ circuits by substituting all nodes with every possible combination of the 
6 gates we are using. Remember, this has to be done for all possible DAGs, the number of which grows super-exponentially with $n$.
Therefore, heavy optimization is needed to be able to complete a search in a reasonable amount of time.

\subsection {Mathematically optimizing the search}

There are $2^{2^4}$ possible 4-input, 1-output boolean functions. However, some of these are equivalent to each other under
permutation of input, which we can achieve at zero cost by just connecting the inputs to our circuit in a different way. Hence,
we can find any circuit that implements a function that is equivalent to the one we want under permutation of inputs. Let 
$P(f)$ be the lexicographically smallest truth table of a function equivalent to $f$ under permutation of inputs. All functions
that produce the same $P(f)$ are said to belong to a $P$-equivalence class. By exhaustively listing the $P$ of all  $2^{2^4}$ 
4-input boolean functions I've found that there are 3984 $P$-equivalence classes. I've verified the correctness of this by
reading some random paper I found on the internet (Correia, Vinícius P., and André I. Reis. "Classifying n-input Boolean functions." Proc. IWS 2001 (2001).)

\subsection {Actually conducting the search}

After generating all the possible n-gate circuits for $n <= 6$, and adding a few cherry-picked 8-gate circuits,
I evaluate each of them on all possible input combinations by using some clever
bit tricks taught in this course (doing bitwise operations on integers instead of using boolean operators, and taking
$C_1 = \text{0x00FF}, C_0 = \text{0x0F0F}, A = \text{0x3333}, B = \text{0x5555}$ to obtain all output combinations). Then I find the P of each circuit, and
find the lowest cost circuit in each of the $P$-equivalence classes which were found.

Up untill now I haven't generated any circuits using inverters. Now, having found the optimal circuit for all $P$-equivalence 
classes, I can test all 16 possible input-inversion combinations for the lowest-cost circuits implementing them, and check if any of 
the produced functions has a smaller cost than the previously-known lowest-cost representative of its P-equivalence class.

In the end, to obtain the final result I find the $P$ of the function I was supposed to implement, and look up the lowest
cost circuit implementing one of the permutation-equivalent functions, which in this case is:

\begin{center}

\begin{tikzpicture}

    \graph[
        layered layout,
        grow=right, 
        component direction=down,
        level distance=2.5cm,
        tail anchor=0,
        head anchor=180,
    ] {
        { [same layer] C1, C0, A, B };

        %GRAPH
    };

\end{tikzpicture}

\end{center}

With a total cost of TOTALCOST. By the way, I'm sorry if the graph looks a bit weird -- it is automatically generated
and I couldn't find a way to make the connections orthogonal. In fact, all of this document is automatically
generated by the same haskell program which conducts the search, parametrized only by the unique nubmer
assigned to me in the supplied .csv file. This means that a pdf with a close-to-optimal solution to the
whole coursework can be generated for any student from the cohort in a matter of seconds. Ain't that cool?
In fact, I went ahead and attached as an appendix the pdf generated for the unique number used as an example in the specification -- 125.
In the example given the lowest cost circuit found had a cost of 60. 
In the solution I automatically generated the cost is 36, which is 40\% less, and uses only 6 gates instead of 11.

\end{document}
