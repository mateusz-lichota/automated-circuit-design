\documentclass[12pt]{article}
\usepackage[left=3cm,right=3cm,top=2cm,bottom=2cm]{geometry} % page settings
\usepackage{amsmath} % provides many mathematical environments & tools
\newtheorem{theorem}{Theorem}
\usepackage{amsfonts}
\usepackage{amssymb}
\usepackage{titlesec}

\usepackage{tikz}
\usetikzlibrary{graphdrawing,graphs} 
\usegdlibrary {layered}

\usetikzlibrary{arrows,shapes.gates.logic.US,shapes.gates.logic.IEC,calc}

\tikzset{
  -|-/.style={
    to path={
      (\tikztostart) -| ($(\tikztostart)!#1!(\tikztotarget)$) |- (\tikztotarget)
      \tikztonodes
    }
  }
}


\begin{document}


\title{\large \textbf{Introduction to Computer Systems Coursework 2 \\ 
\vspace{0.5em}
\large Combinatorial Circuit Design}}


\author{Mateusz Lichota}
\date{\today}
\maketitle

As per the list supplied on CATe my number is $204_{10}$, which is $3030$ in base four.

Looking it up in the table I obtained the following specification for my circuit:

% spec table here

From this specification a truth table can be constructed:

% truth table here

The corresponding Karnaugh map is:

% karnaugh map here


\begin{tikzpicture}

    \graph[
        layered layout,
        grow=right, 
        component direction=down,
        level distance=2cm,
        component order=decreasing node number,
    ] {
        { [same layer] C1, C0, A, B };

        %GRAPH
    };

\end{tikzpicture}

\end{document}
